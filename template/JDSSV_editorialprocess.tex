\documentclass[article]{jdssv}

%% -- LaTeX packages and custom commands ---------------------------------------

%% recommended packages
\usepackage{thumbpdf,lmodern}
\usepackage{bm}
\usepackage{todonotes} %% Remove this package when finishing the manuscript


%% another package (only for this demo article)
\usepackage{framed}

%% new custom commands
\newcommand{\class}[1]{`\code{#1}'}
\newcommand{\fct}[1]{\code{#1()}}
\newcommand{\ma}[1]{\ensuremath{\mathbf{#1}}}


%% -- Article metainformation (author, title, ...) -----------------------------

%% - \author{} with primary affiliation
%% - \Plainauthor{} without affiliations
%% - Separate authors by \And or \AND (in \author) or by comma (in \Plainauthor).
%% - \AND starts a new line, \And does not.
\author{Patrick J.F. Groenen\\Erasmus University Rotterdam
   \And Stefan Van Aelst\\KU Leuven}
\Plainauthor{Patrick J.F. Groenen, Stefan Van Aelst}

%% - \title{} in title case
%% - \Plaintitle{} without LaTeX markup (if any)
%% - \Shorttitle{} with LaTeX markup (if any), used as running title
\title{Guidelines for the Journal of Data Science, Statistics, and Visualisation}
\Plaintitle{Guidelines for the Journal of Data Science, Statistics, and Visualisation Version 0.2}
\Shorttitle{JDSSV Guidelines, Version 0.2}

%% - \Abstract{} almost as usual
\Abstract{
  This short article illustrates how to write a manuscript for the   \emph{Journal of Data Science, Statistics and Visualisation} (JDSSV) using its {\LaTeX} style files.   Please follow JDSSV's style guidelines precisely. Also, it is recommended to keep the {\LaTeX} code as simple as possible, that is, avoid inclusion of packages/commands that are not necessary. 
}

%% - \Keywords{} with LaTeX markup, at least one required
%% - \Plainkeywords{} without LaTeX markup (if necessary)
%% - Should be comma-separated and in sentence case.
\Keywords{JDSSV, style guidelines, comma-separated, not capitalized, \proglang{R}}
\Plainkeywords{JDSSV, style guidelines, comma-separated, not capitalized, R}

%% - \Address{} of at least one author
%% - May contain multiple affiliations for each author
%%   (in extra lines, separated by \emph{and}\\).
%% - May contain multiple authors for the same affiliation
%%   (in the same first line, separated by comma).
\Address{
  Patrick J.F. Groenen\\
  Econometric Institute\\
  Erasmus School of Economics\\
  Erasmus University Rotterdam\\
  P.O. Box  1738 \\
  3000 DR Rotterdam, The Netherlands\\
  E-mail: \email{groenen@ese.eur.nl}\\
  URL: \url{https://personal.eur.nl/groenen/}\\
\ \\
%}
%
%\Address{
  Stefan Van Aelst\\
  Department of Mathematics\\
  KU Leuven\\
  Celestijnenlaan 200B\\
  3001 Leuven, Belgium\\
  E-mail: \email{Stefan.VanAelst@kuleuven.be}\\
  URL: \url{https://wis.kuleuven.be/stat/robust}
}


\begin{document}


%% -- Introduction -------------------------------------------------------------

%% - In principle "as usual".
%% - But should typically have some discussion of both _software_ and _methods_.
%% - Use \proglang{}, \pkg{}, and \code{} markup throughout the manuscript.
%% - If such markup is in (sub)section titles, a plain text version has to be
%%   added as well.
%% - All software mentioned should be properly \cite-d.
%% - All abbreviations should be introduced.
%% - Unless the expansions of abbreviations are proper names (like "Journal
%%   of Statistical Software" above) they should be in sentence case (like
%%   "generalized linear models" below).

\section{Mission}

JDSSV\footnote{This document is an adaptation of the style guide of the Journal of Statistical Software \citep{jss_style_guide}.} is an international refereed journal that creates a forum to present recent progress and ideas in the different disciplines of data science, statistics, and visualisation. It welcomes contributions to data science, statistics, and visualisation, and in particular those aspects which link and integrate these subject areas. Articles can cover topics such as machine learning and statistical learning, the visualisation and verbalisation of data, visual analytics, big data infrastructures and analytics, interactive learning, and advanced computing. Papers that discuss two or more research areas of the journal are favoured. Scientific contributions should be of a high standard. Articles should be oriented towards a wide scientific audience of statisticians, data scientists, computer scientists, data analysts, etc.

The journal welcomes original contributions that are not being considered for publication elsewhere and contain a high level of novelty. Papers with a thorough but concise review of a certain topic with the potential to provide new insights are also welcome. Manuscripts submitted to the journal generally are accompanied by supplementary material that containing software code, technical derivations or detailed explanations, additional examples, etc. All submitted material will be reviewed by the assigned associate editor and reviewers of the manuscript. 

Manuscripts may have a substantial theoretical component, but it is expected that all manuscripts contain at least one application on empirical or simulated data. The journal emphasizes the reproducibility of the results presented its papers. Therefore, all data and software code that is necessary to reproduce the empirical results in the manuscript should be made available in a user friendly manner. If the empirical data cannot be released for reasons of confidentiality or otherwise, then a generated dataset with comparable properties should be provided. 



\section[]{Preparing your Manuscript for Submission}

All submissions to JDSSV should be written in {\LaTeX} using the JDSSV style files provided on the journal website \url{https://jdssv.org}. For initial submission it suffices providing the pdf version of the manuscript together with all the necessary supplementary material including software code to reproduce all results in the manuscript. The final version of accepted manuscripts should adhere to all the style guidelines and should incorporate all changes requested by the production editor. All {\LaTeX}  source files of the final manuscript should be submitted and accepted manuscripts are only published if these files comply with all guidelines and instructions provided by the journal. 



\section[]{Software}

The journal expects that submissions contain accompanying
software with the aim of reproducibility of the results and application of the proposed methodology to other data by the reader.  All existing software used in the paper should be properly referenced. Code should be delivered in an easily readable manner with clear instructions on its use, and preferably is accompanied by instructive examples of its use. We highly recommend to provide code that can be used in open-source software such as \proglang{R} \citep{R}, \proglang{Python} \citep{python}, \proglang{Julia} \citep{Julia}, \proglang{Octave} \citep{octave}, etc. 

To make code widely accessible, we advise making it available in a repository such as \url{https://zenodo.org} where it will receive a permanent Digital Object Identifier (DOI) which can be included in the manuscript. 
%\todo{\tiny Alternatives: (1) We store the data/code on our JDSSV website with the article. (2) We upload it on Zenodo ourselves (not sure if that can be done).}  

The provided code should at least consist of a file or set of files for the functions that execute the core of the method. For reproducibility, another necessary file is the script that creates the results (tables and figures) of the paper. For methods that run very long, provide a toy example that runs sufficiently fast and highlights the properties of the method. Nonproprietary data should be provided or a permanent link to these data should be given. The corresponding script to analyze the data should read these data. 

\begin{sloppypar}
For increased readability, please apply the following naming conventions for code: Start function names with a verb, e.g., \code{set_initialisation()}, \code{compute_loss()}, \code{update_X()}, etc. when appropriate. Give  objects descriptive names or closely follow the notation in the paper.
For programming conventions (particularly in \proglang{R}), see Hadley Wickham's guidelines (\url{http://r-pkgs.had.co.nz/style.html}) and the use of the \pkg{styler} package is recommended. For \proglang{python}, consider Google's recommendations (\url{https://github.com/google/styleguide/blob/gh-pages/pyguide.md}) Sufficient comments should be added to the code to make it understandable. For \proglang{Octave} and \proglang{MatLab} \citep{matlab}, consider Richard Johnson's MatLab Style Guidelines 2.0 (\url{https://www.mathworks.com/matlabcentral/fileexchange/46056-matlab-style-guidelines-2-0}).
\end{sloppypar}

For writing about software JDSSV requires authors to use the markup
\verb|\proglang{}| (programming languages and large programmable systems), \verb|\pkg{}| (software packages), \verb|\code{}| (functions, commands, arguments, etc.). If there is such markup in (sub)section titles (as above), a plain text version has to be provided in the {\LaTeX} command as well. Below we also illustrate how abbrevations should be introduced and citation commands can be employed. See the {\LaTeX} code for more details.

\section{Review process}

All manuscripts should be submitted online at  \url{https://jdssv.org}. JDSSV uses a single blind review process. Upon submission of their manuscript, authors have the opportunity to provide a short list of suggested reviewers as well as a few names of researchers that preferably should not be contacted for reviewing the manuscript.  All submitted manuscript will undergo automatic checking for plagiarism and will not be considered for further review in case of plagiarism. The manuscript will be assigned to one of the editors who will make an initial screening to check the quality of the submitted work, possibly with the help of an associate editor. After positive screening, the manuscript will be assigned to an associate editor who will seek the opinion of at least two reviewers. Reviewers are asked to send their report within one month. Associate editors should typically make their recommendation one week after reception of the review reports. The initial review process should normally not take more than three months. All communication between the authors and the journal will be administered by the journal editors.  


%% -- Manuscript ---------------------------------------------------------------

%% - In principle "as usual" again.
%% - When using equations (e.g., {equation}, {eqnarray}, {align}, etc.
%%   avoid empty lines before and after the equation (which would signal a new
%%   paragraph.
%% - When describing longer chunks of code that are _not_ meant for execution
%%   (e.g., a function synopsis or list of arguments), the environment {Code}
%%   is recommended. Alternatively, a plain {verbatim} can also be used.
%%   (For executed code see the next section.)

\section{Correspondence Analysis} \label{sec:models}

As an example, we provide a simple implementation of correspondence analysis \citep[see, e.g.,][]{greenacre2010biplots}. For the analysis of bivariate categorical data or other tables that contain counts, correspondence analysis can be a useful technique to visualize important relations between the categories. Consider word count data on explanations of data science by several websources provided by \citet{Lubbe2018}, see Table~\ref{tab:word_counts}.

\begin{table}
\caption{Part of the table with word counts describing data science by different web sources as collected by \citet{Lubbe2018}.}
\label{tab:word_counts}
\begin{center}
\begin{tabular}{lccccccc} \hline
            & dsc.test & dsctechniques & datadiversity & dsc.tips & $\ldots$ & wikipedia \\ \hline
statistics  &       6  &           2   &          2    &    0     & $\ldots$ &  2 \\
big data    &       2  &           0   &          3    &    5     & $\ldots$ &  0 \\
analytics   &       1  &           0   &          2    &    0     & $\ldots$ &  2 \\
database    &       2  &           0   &          0    &    2     & $\ldots$ &  1 \\
insight     &       0  &           0   &          4    &    0     & $\ldots$ &  2 \\
$\vdots$    & $\vdots$ &    $\vdots$   &   $\vdots$    & $\vdots$ &$ \ddots$ & $\vdots$ \\
programming &       7  &           0   &          0    &    0     & $\ldots$ &  0 \\
\hline
\end{tabular}
\end{center}
\end{table}

Let $\ma{F}$ be the matrix containing the values in Table~\ref{tab:word_counts}. The most simple model to fit such a table is the independence model in matrix $\ma{E}$ defined as 
\begin{eqnarray*}
  \ma{E} = n^{-1}\ma{D}_r \ma{11}^\top \ma{D}_c
\end{eqnarray*}
with $\ma{1}$ a vector of ones of appropriate length, $\ma{D}_r = \textrm{Diag}(\ma{F1})$ the diagonal matrix with row sums of $\ma{F}$, $\ma{D}_c = \textrm{Diag}(\ma{F^\top1})$ the diagonal matrix with column sums of $\ma{F}$, and $n = \ma{1^\top F1}$ be the sum of all elements in $\ma{F}$. The goal of correspondence analysis is to find a matrices of row and column coordinates $\ma{R}$ and $\ma{C}$ such that
\begin{eqnarray*}
  L(\ma{R}, \ma{C}) = \|\ma{D}_r^{-1/2} (\ma{F} - \ma{E} - \ma{D}_r\ma{RC}\ma{D}_c^\top) \ma{D}_c^{-1/2}\|^2
\end{eqnarray*}
is minimized \citep[see, for example,][]{vandeveldenetal2009seriation}. It may be verified that the least-squares optimal solution is obtained for by computing the singular value decomposition (SVD) 
\begin{eqnarray*}
  \ma{D}_r^{-1/2} (\ma{F} - \ma{E})\ma{D}_c^{-1/2} = \ma{UDV}^\top
\end{eqnarray*}
and computing
\begin{eqnarray*}
  \ma{R} &=& n^{1/2}\ma{D}_r^{-1/2} \ma{UD}^{\alpha}\\
  \ma{C} &=& n^{1/2}\ma{D}_c^{-1/2} \ma{VD}^{1-\alpha}
\end{eqnarray*}
for some $\alpha$. Note that the least-squares optimal rank $p$ solution is obtained by taking the first $p$ columns of $\ma{R}$ and $\ma{C}$. If $\alpha = 1$ the so-called row principal solution is obtained where the row points are the weighted centroids of the column points, $\alpha = 0$ gives the column principal solution with column points being the weighted average of the row points, and $\alpha = 1/2$ results in the symmetric solution. The importance measure of the dimensions are given by the inertia, that is, the diagonal elements of $\ma{D}^2$. For more details, see, for example, \cite{greenacre2010biplots}.

A simple implementation of correspondence analysis in \proglang{R} is given by the function \code{corana()} given by
\begin{Code}
corana <- function(dat, alpha = 0.5){
  # Perform correspondence analysis
  # Input: 
  ##  dat   numeric matrix dat with nonnegative entries
  ##  alpha = 1 is row principal standardisation
  ##        = 0 is column principle standardisation
  ##        = 0.5 is symmetric standardisation
  dat <- as.matrix(dat)
  Dr <- rowSums(dat)
  Dc <- colSums(dat)
  n  <- sum(Dr)
  # Compute expected values under the independence model
  E  <- outer(Dr, Dc)/n
  tt <- svd(diag(Dr^-0.5) %*% (dat - E) %*% diag(Dc^-0.5))
  ## Remove dimensions with singular value zero
  ind <- tt$d > 1e-10
  tt$d <- tt$d[ind]
  tt$u <- tt$u[, ind]
  tt$v <- tt$v[, ind]
  ## Compute row scores R and column scores C 
  R  <- n^0.5 * diag(Dr^-0.5) %*% tt$u %*% diag(tt$d^alpha)
  C  <- n^0.5 * diag(Dc^-0.5) %*% tt$v %*% diag(tt$d^(1 - alpha))
  rownames(R) <- rownames(dat)
  rownames(C) <- colnames(dat)
  ## Compute relative contribution to inertia per dimension 
  row.inert   <- outer(Dr/n, tt$d^(2*alpha),       "/") * R^2
  col.inert   <- outer(Dc/n, tt$d^(2*(1 - alpha)), "/") * C^2
  ## Compute reconstructed Chi-square distance for row and column points
  row.dist    <- R^2 %*% diag(tt$d^(2 - 2*alpha))
  row.dist    <- row.dist / outer(rowSums(row.dist), rep(1, ncol(R))) 
  col.dist    <- C^2 %*% diag(tt$d^(2 - 2*(1 - alpha)))
  col.dist    <- col.dist / outer(rowSums(col.dist), rep(1, ncol(C))) 
  ## Prepare list of results
  out <- list(R = R, C = C, sing.val = tt$d, 
              row.inert = row.inert, col.inert = col.inert,
              row.dist  = row.dist,  col.dist  = col.dist)
  class(out) <- "corana"
  return(out)
}\end{Code}
A plot is given by the \code{plot.corana()} method
\begin{Code}
plot.corana <- function(out, dims = 1:2, ...){
  R <- out$R[, dims]
  C <- out$C[, dims]
  ## Set up coordinate system
  coord <- rbind(R, C)
  plot(coord[, 1], coord[, 2], type = "n", asp = 1, las = 1,
       xlab = paste0("Dim ", dims[1]), ylab = paste0("Dim ", dims[2]), ...)
  abline(h = 0, v = 0, col = "gray")
  ## Use reconstructed distance as importance measure for transparency and size
  row.alpha <- rowSums(out$row.dist[, dims])
  row.alpha <- (row.alpha/max(row.alpha))^0.7
  col.alpha <- rowSums(out$col.dist[, dims])
  col.alpha <- (col.alpha/max(col.alpha))^0.7
  ## Plot row and column points 
  points(R[, 1], R[, 2], pch = 20, col = rgb(1, 0, 0, row.alpha))
  points(C[, 1], C[, 2], pch = 20, col = rgb(0, 0, 1, col.alpha))
  ## Write text labels
  text(R[, 1], R[, 2], rownames(R), pos = compute.pos(R), cex = 2*row.alpha,
       col = rgb(1, 0, 0, ifelse(row.alpha > 0.1, row.alpha, 0)))
  text(C[, 1], C[, 2], rownames(C), pos = compute.pos(R), cex = 2*col.alpha,
       col = rgb(0, 0, 1, ifelse(col.alpha > 0.1, col.alpha, 0)))
}
\end{Code}
Assuming that \code{ds.word.cnt} contains the full matrix of which a part is shown in Table~\ref*{tab:word_counts}, then the following code does a correspondence analysis on these data:
\begin{Code}
out.corana <- corana(ds.word.cnt)
plot(out.corana, dims = 2:3)
\end{Code}
that yields the plot of Dimensions 2 and 3 in Figure~\ref{fig:ca_biplot_dims23}. In this plot, the points and labels are made more transparent and the labels decrease in size as they are represented worse in these two dimensions.

\begin{figure}
\begin{center}
\includegraphics[width=10cm]{ca_example.pdf}
\end{center}
\caption{Biplot of a correspondence analysis on word counts used in descriptions of data science by different web sources in dimensions 2 and 3.} \label{fig:ca_biplot_dims23}
\end{figure}







%% -- Summary/conclusions/discussion -------------------------------------------

%\section{Summary and discussion} \label{sec:summary}
%
%\begin{leftbar}
%As usual \dots
%\end{leftbar}


%% -- Optional special unnumbered sections -------------------------------------
\newpage
\section*{Computational Details}

If necessary or useful, information about certain computational details
such as version numbers, operating systems, or compilers could be included
in an unnumbered section. Also, auxiliary packages (say, for visualizations,
maps, tables, \dots) that are not cited in the main text can be credited here.


The results in this paper were obtained using
\proglang{R}~3.5.1. \proglang{R} itself
and all packages used are available from the Comprehensive
\proglang{R} Archive Network (CRAN) at
\url{https://CRAN.R-project.org/}.


\section*{Acknowledgments}

All acknowledgments should be collected in this
unnumbered section before the references. It may contain the usual information
about funding and feedback from colleagues/reviewers/etc. Furthermore,
information such as relative contributions of the authors may be added here
(if any).


%% -- Bibliography -------------------------------------------------------------
%% - References need to be provided in a .bib BibTeX database.
%% - All references should be made with \cite, \citet, \citep, \citealp etc.
%%   (and never hard-coded). See the FAQ for details.
%% - JSS-specific markup (\proglang, \pkg, \code) should be used in the .bib.
%% - Titles in the .bib should be in title case.
%% - DOIs should be included where available.

\bibliography{refs}


%% -- Appendix (if any) --------------------------------------------------------
%% - After the bibliography with page break.
%% - With proper section titles and _not_ just "Appendix".

\newpage

\begin{appendix}

\section{More Technical Details} \label{app:technical}


Appendices can be included after the bibliography (with a page break). Each section within the appendix should have a proper section title (rather than just \emph{Appendix}).

%For more technical style details, please check out JSS's style FAQ at
%\url{https://www.jstatsoft.org/pages/view/style#frequently-asked-questions}
%which includes the following topics:
%\begin{itemize}
%  \item All main words in titles and headers start with a capital.
%  \item Use vectorized graphics formats such as eps and pdf when possible. Graphics should be of high quality while trying to minimize the file size.
%\end{itemize}


\section[Using BibTeX]{Using \textsc{Bib}{\TeX}} \label{app:bibtex}

References need to be provided in a \textsc{Bib}{\TeX} file (\code{.bib}). All references should be made with \verb|\cite|, \verb|\citet|, \verb|\citep|, \verb|\citealp| etc.\ (and never hard-coded). This commands yield different formats of author-year citations and allow to include additional details (e.g., pages, chapters, \dots) in brackets. 
%In case you are not familiar with these commands see the JSS style FAQ for details.

Cleaning up \textsc{Bib}{\TeX} files is a somewhat tedious task -- especially when acquiring the entries automatically from mixed online sources. However, it is important that informations are complete and presented in a consistent style to avoid confusions. JDSSV requires the following format.
\begin{itemize}
  \item Specific markup (\verb|\proglang|, \verb|\pkg|, \verb|\code|) should
    be used in the references.
  \item Titles should be inserted in title case.
  \item Journal titles should not be abbreviated and in title case.
  \item DOIs should be included where available.
  \item Software should be properly cited as well. For \proglang{R} packages
    \code{citation("pkgname")} typically provides a good starting point.
\end{itemize}
\end{appendix}
\newpage
%% -----------------------------------------------------------------------------


\end{document}
